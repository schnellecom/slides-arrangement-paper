\documentclass{beamer}
\usepackage{graphicx} % Required for inserting images

\usepackage[utf8]{inputenc}
% \usepackage[T1]{fontenc}
\usepackage{lmodern}
\usepackage{amsmath,amssymb}
\usepackage{microtype}
\usepackage{ellipsis}
%\usepackage[ngerman]{babel}
\let\openbox\undefined
\usepackage{mathtools}
% \usepackage{enumitem}
\let\openbox\undefined
\usepackage{amsthm}
\usepackage{thmtools}
\usepackage{graphicx}
\usepackage{stmaryrd}
\usepackage{tikz}
\usetikzlibrary{positioning}
\usepackage{algpseudocode}
\usepackage[absolute,overlay]{textpos}
\usepackage{url}
\usepackage[
backend=biber,
style=numeric,
]{biblatex}
\addbibresource{./references.bib}
\usepackage[normalem]{ulem}
\usepackage{xcolor}

\declaretheoremstyle[
    headpunct=,
    spacebelow=2em,
    spaceabove=1em,
    postheadspace=\newline,
    ]{aufgabe}
\declaretheorem[style=aufgabe]{aufgabe}
\numberwithin{equation}{aufgabe}
\addtolength\jot{1ex}

\newtheorem{proposition}{Proposition}
\renewcommand\qedsymbol{$\square$}

\newcommand\R{\mathbb R}
\newcommand\Z{\mathbb Z}
\newcommand\N{\mathbb N}
\newcommand\C{\mathbb C}
\newcommand{\Q}{\mathbb Q}
\newcommand{\F}{\mathbb{F}}
\newcommand{\ass}{\underline{Assume:}  }
\newcommand{\zz}{\underline{t.s.:}  }

\usetheme[compress]{Berlin}
\setbeamertemplate{footline}[frame number]{}
\setbeamertemplate{navigation symbols}{}
\setbeamertemplate{footline}{}

\makeatletter
\beamer@theme@subsectionfalse%
\makeatother

\title{Mechanical Comparison of Arrangement Strategies for Topological Interlocking Assemblies}
\author{Lukas Schnelle}
\date{Dec 2023}

\begin{document}

\maketitle

% \section{Problem statement}
% \begin{frame}{An Example}
%     \visible<2-4>{
%     Let $\forall i \in [n]: a_i \in \R^1, b_i \in \R $ some data.\\
%     \textbf{Goal:} Find $x \in \R$ s.th. $\only<3-4>{\textcolor{red}{min_{x\in \R}(ax - b)}} \only<2>{ax = -b}$}
%     \only<1-3>{
%     \begin{center}
%         \includegraphics[width=0.6\textwidth]{images/pointplot.png}
%     \end{center}
%     }
%     \only<4>{
%     \begin{center}
%         \includegraphics[width=0.6\textwidth]{images/lineplot.png}
%     \end{center}
%     }
% \end{frame}

\section{Mathematics}
\begin{frame}{Topological Interlocking}
    \begin{definition}[Topological Interlocking]
        A topological interlocking assembly can be defined as an arrangement of blocks that are in contact with each other together with a frame such that, if the frame is fixed, any non-empty finite subset of blocks of the arrangement is prevented from moving.
    \end{definition}
    \pause
    \textbf{Here:} 
    \begin{itemize}
        \item planar topological interlocking assemblies, i.e. between two parallel planes in 3D-space,
        \pause\item use perimeter as the frame,
        \pause\item only copies of the same block differently arranged.
    \end{itemize}  
\end{frame}
\begin{frame}{The Versatile Block}
    The Versatile Block is a polyhedron embedded in $\R^3$, given by vertices $\{v_1,\ldots,v_9 \, \}$, edges
    \begin{align*}
        \{  &\{ v_1, v_2 \},  \{ v_1, v_3 \},  \{ v_1, v_4 \}, \{ v_1, v_5 \}, \{ v_1, v_9 \},  \{ v_2, v_3 \}, \{ v_2, v_5 \}, \\
        & \{ v_2, v_6 \},  \{ v_2, v_7 \},  \{ v_3, v_4 \},  \{ v_3, v_7 \},  \{ v_4, v_7 \}, \{ v_4, v_8 \}, \{ v_4, v_9 \},\\
        & \{ v_5, v_6 \}, \{ v_5, v_7 \},  \{ v_5, v_9 \}, \{ v_6, v_7 \},  \{ v_7, v_8 \},  \{ v_7, v_9 \},  \{ v_8, v_9 \}  \},
    \end{align*} 
    and triangular faces
    \begin{align*}
        \{& \{v_1, v_2, v_3\}, \{v_1, v_2, v_5\},  \{v_1, v_3, v_4\}, \{v_1, v_4, v_9\}, \{v_1, v_5, v_9\}, \{v_2, v_3, v_7\},  \{v_2, v_5, v_6\},\\
        &\{v_2, v_6, v_7\},  \{v_3, v_4, v_7\},  \{v_4, v_7, v_8\},  \{v_4, v_8, v_9\},  \{v_5, v_6, v_7\},  \{v_5, v_7, v_9\}, \{v_7, v_8, v_9\}\}
    \end{align*}
    together with coordinates
    \begin{alignat*}{3}
    &v_1 = (0, 0, 0) , v_2 = (1, 1, 0) ,  v_3 = (2, 0, 0) , v_4 = (1, -1, 0) , \\
    &v_5 = (0, 1, 1) ,  v_6 = (1, 1, 1) ,  v_7 = (1, 0, 1) , v_8 = (1, -1, 1) , v_9 = (0, -1, 1).
    \end{alignat*} 
\end{frame}
\begin{frame}{Wallpaper Groups}

\end{frame}
\begin{frame}{Planar Assemblies of the Versatile Block}

\end{frame}

\section{Mechanics and Simulation}
\begin{frame}{Finite Element Method}
    
\end{frame}

\begin{frame}{Problem formulation}

\end{frame}

\begin{frame}{Simulation Setup}
    
\end{frame}

\subsection{Simulation Results}
\begin{frame}{Stresses}
    
\end{frame}

\section{Interlocking Flows}
\begin{frame}{Combinatorial Method}
    
\end{frame}

\begin{frame}{Combinatorial Results}
    
\end{frame}

\section{Outlook}

\begin{frame}
    
\end{frame}

\appendix
\begin{frame}    
\printbibliography 
\end{frame}

\end{document}
